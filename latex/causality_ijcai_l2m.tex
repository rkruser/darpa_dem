% IJCAI / L2M research summary

% https://ijcai20.org/call-for-papers.html

% These are the instructions for authors for IJCAI-20.

\documentclass{article}
\pdfpagewidth=8.5in
\pdfpageheight=11in
% The file ijcai20.sty is NOT the same than previous years'
\usepackage{ijcai20}

% Use the postscript times font!
\usepackage{times}
\usepackage{soul}
\usepackage{url}
\usepackage[hidelinks]{hyperref}
\usepackage[utf8]{inputenc}
\usepackage[small]{caption}
\usepackage{graphicx}
\usepackage{amsmath}
\usepackage{amsthm}
\usepackage{booktabs}
\usepackage{algorithm}
\usepackage{algorithmic}
\urlstyle{same}

% the following package is optional:
%\usepackage{latexsym} 

% See https://www.overleaf.com/learn/latex/theorems_and_proofs
% for a nice explanation of how to define new theorems, but keep
% in mind that the amsthm package is already included in this
% template and that you must *not* alter the styling.
%\newtheorem{example}{Example}
%\newtheorem{theorem}{Theorem}

% Following comment is from ijcai97-submit.tex:
% The preparation of these files was supported by Schlumberger Palo Alto
% Research, AT\&T Bell Laboratories, and Morgan Kaufmann Publishers.
% Shirley Jowell, of Morgan Kaufmann Publishers, and Peter F.
% Patel-Schneider, of AT\&T Bell Laboratories collaborated on their
% preparation.

% These instructions can be modified and used in other conferences as long
% as credit to the authors and supporting agencies is retained, this notice
% is not changed, and further modification or reuse is not restricted.
% Neither Shirley Jowell nor Peter F. Patel-Schneider can be listed as
% contacts for providing assistance without their prior permission.

% To use for other conferences, change references to files and the
% conference appropriate and use other authors, contacts, publishers, and
% organizations.
% Also change the deadline and address for returning papers and the length and
% page charge instructions.
% Put where the files are available in the appropriate places.

\title{Learning to Predict Based on Causality Using an Adversarial Objective}

% Single author syntax
%\author{
%    Christian Bessiere
%    \affiliations
%    CNRS, University of Montpellier, France
%    \emails
%    pcchair@ijcai20.org
%}

% Multiple author syntax (remove the single-author syntax above and the \iffalse ... \fi here)
% Check the ijcai20-multiauthor.tex file for detailed instructions
%\iffalse
\author{
Ryen Krusinga$^1$
\and
David Jacobs$^2$
\affiliations
$^1$University of Maryland\\
$^2$University of Maryland
\emails
\{krusinga, djacobs\}@umiacs.umd.com
}
%\fi

\begin{document}

\maketitle

\begin{abstract}
In a given learning environment, some variables are causal, and some are merely correlated with the prediction targets. Standard machine learning models make no distinction between the two, making the models less robust to domain shift, in which causal factors remain invariant but correlations change. Given background knowledge about the causal structure of the environment, we demonstrate a simple adversarial method to train a predictor that ignores non-causal information. We show results on two artificial environments.
\end{abstract}

This is a test citation \cite{Oh2015}.


















%%%%% Training category proportions %%%%%%%
% Paddle size \textbackslash Background Color
\begin{table}
\centering
\begin{tabular}{r | c | c}
 & Blue & Red \\
\hline
Large & 0.095 & 0.422 \\
\hline
Small & 0.393 & 0.090
\end{tabular}
\caption{Training data category proportions among 2458 frames from 50 different games. Left column: large vs. small paddle size compared to the opponent agent. Top row: background color of the game. Generated from 20 games each with large paddle / red background and small paddle / blue background; 5 games each in the other two quadrants.}
\label{tab:exp1_prop}
\end{table}
%%%%%%%%%%%%


%%%%% Training win proportions %%%%%%%
% Paddle size \textbackslash Background Color
\begin{table}
\centering
\begin{tabular}{r | c | c}
 & Blue & Red \\
\hline
Large & 0.282 & 0.553 \\
\hline
Small & 0.074 & 0.000
\end{tabular}
\caption{Training data win proportions among each combination of attributes. Each frame is annotated with the win condition of the game it is sampled from. Over half the frames containing a large paddle with red background were in a winning game. Although the background color does not affect gameplay, a very different proportion of frames with the large paddle ad blue background were in a winning game, mainly due to small sample size. Small sample size also explains why the lower right corner is zero.}
\label{tab:exp1_win}
\end{table}
%%%%%%%%%%%%

%%%%% Test category proportions %%%%%%%
% Paddle size \textbackslash Background Color
\begin{table}
\centering
\begin{tabular}{r | c | c}
 & Blue & Red \\
\hline
Large & 0.079 & 0.376 \\
\hline
Small & 0.465 & 0.081
\end{tabular}
\caption{Test data proportions among 2726 frames. Compare to Table \ref{tab:exp1_prop}}
\label{tab:exp1_test_prop}
\end{table}
%%%%%%%%%%%%


%%%%% Test win proportions %%%%%%%
% Paddle size \textbackslash Background Color
\begin{table}
\centering
\begin{tabular}{r | c | c}
 & Blue & Red \\
\hline
Large & 0.594 & 0.381 \\
\hline
Small & 0.000 & 0.000
\end{tabular}
\caption{Test data win proportions. Compare to Table \ref{tab:exp1_win}}
\label{tab:exp1_test_win}
\end{table}
%%%%%%

%%%%% Learned train proportions %%%%%%%
% Paddle size \textbackslash Background Color
\begin{table}
\centering
\begin{tabular}{r | c | c}
 & Blue & Red \\
\hline
Large & 0.263 & 0.556 \\
\hline
Small & 0.072 & 0.000
\end{tabular}
\caption{Average win probabilities predicted by a basic neural net trained on the training data.}
\label{tab:exp1_model}
\end{table}
%%%%%%%%%%%%%%%%%%

%%%%% Learned test proportions %%%%%%%
% Paddle size \textbackslash Background Color
\begin{table}
\centering
\begin{tabular}{r | c | c}
 & Blue & Red \\
\hline
Large & 0.277 & 0.627 \\
\hline
Small & 0.030 & 0.000
\end{tabular}
\caption{Average win probabilities predicted by a basic neural network on the test data. Note the proportions are more similar to the training data than to the actual test data proportions, as expected, since there is very little information in each frame to indicate a win besides color and paddle size; the network mainly defaults to its priors in those quadrants.}
\label{tab:exp1_test_win}
\end{table}
%%%%%%%%%%%%

%%%%% Intervention category proportions %%%%%%%
% Paddle size \textbackslash Background Color
\begin{table}
\centering
\begin{tabular}{r | c | c}
 & Blue & Red \\
\hline
Large & 0.084 & 0.426 \\
\hline
Small & 0.105 & 0.385
\end{tabular}
\caption{Intervention data category proportions among 2557 frames in an interventional dataset.}
\label{tab:exp1_int_prop}
\end{table}
%%%%%%%%%%%%


%%%%% Intrevention win proportions %%%%%%%
% Paddle size \textbackslash Background Color
\begin{table}
\centering
\begin{tabular}{r | c | c}
 & Blue & Red \\
\hline
Large & 0.594 & 0.605 \\
\hline
Small & 0.000 & 0.000
\end{tabular}
\caption{Intervention data win proportions.}
\label{tab:exp1_int_win}
\end{table}
%%%%%%%%%%%%

%%%%% Intrevention model test proportions %%%%%%%
% Paddle size \textbackslash Background Color
\begin{table}
\centering
\begin{tabular}{r | c | c}
 & Blue & Red \\
\hline
Large & 0.276 & 0.613 \\
\hline
Small & 0.051 & 0.000
\end{tabular}
\caption{Win probabilities as predicted by the neural network model.}
\label{tab:exp1_int_model}
\end{table}
%%%%%%%%%%%%

%%%%%%%%%%%%%%%%%%%%%%%%%%%%%%%%%%%%%%%%%%%%%%%%%%%%%%%%%%%%%%%%%%%%
% Text list
% 
% Experimental and model design
% Results with and without adversarial training
% (Fix problem with small features)
% 
% 
% 
%%%%%%%%%%%%%%%%%%%%%%%%%%%%%%%%%%%%%%%%%%%%%%%%%%%%%%%%%%%%%%%%%%%%
%%%%%%%%%%%%%%%%%%%%%%%%%%%%%%%%%%%%%%%%%%%%%%%%%%%%%%%%%%%%%%%%%%%%
% Figure list
%
% Game images!
% 4-quadrant tables giving sample proportions
% Tables giving performance of different models in those quadrants
% Maybe training curves, maybe.
% 
% 
% 
% 
% 
%%%%%%%%%%%%%%%%%%%%%%%%%%%%%%%%%%%%%%%%%%%%%%%%%%%%%%%%%%%%%%%%%%%%




















\bibliographystyle{named}
\bibliography{causality_l2m_bib}

\end{document}
